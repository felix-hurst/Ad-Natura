\documentclass[12pt]{article}

\usepackage{enumitem}

\usepackage{amssymb}
\usepackage{amsfonts}
\usepackage{amsmath}

\usepackage{hyperref}
\hypersetup{colorlinks=true,
    linkcolor=blue,
    citecolor=blue,
    filecolor=blue,
    urlcolor=blue,
    unicode=false}
\urlstyle{same}

\newlist{todolist}{itemize}{2}
\setlist[todolist]{label=$\square$}
\usepackage{pifont}
\newcommand{\cmark}{\ding{51}}%
\newcommand{\xmark}{\ding{55}}%
\newcommand{\done}{\rlap{$\square$}{\raisebox{2pt}{\large\hspace{1pt}\cmark}}%
\hspace{-2.5pt}}
\newcommand{\wontfix}{\rlap{$\square$}{\large\hspace{1pt}\xmark}}

\begin{document}

\title{Final Documentation Checklist}
\author{Spencer Smith}
\date{\today}

\maketitle

% Show an item is done by   \item[\done] Frame the problem
% Show an item will not be fixed by   \item[\wontfix] profit

The following checklist is relevant when finishing a project in the SE capstone
course.

\begin{itemize}
  
\item Clean-up repo
  \begin{todolist}
  \item Repo name is meaningful (not generic name with capstone in it)
  \item Issues are closed (with appropriate explanation and/or changes to the
  repo contents)
  \item Folders for extras that are not used are deleted
  \item Meaningful README file on GitHub landing page
  \item LICENSE is filled in
  \item SRS templates that were not used are deleted
  \end{todolist}

\item Changes are clear to TAs
  \begin{todolist}
  \item TA can see traceability between issue and change, or issue and explanation
  \end{todolist}

\item Project is easy to navigate
  \begin{todolist}
  \item Clear traceability between high-level and low-level design and to code
  \item Clear traceability between test cases and requirements
  \item Clear traceability between test cases and modules/classes
  \end{todolist}

\end{itemize}

\end{document}
