\documentclass{article}

\usepackage{booktabs}
\usepackage{tabularx}

\title{Development Plan\\\progname}

\author{\authname}

\date{}

%% Comments

\usepackage{color}

\newif\ifcomments\commentstrue %displays comments
%\newif\ifcomments\commentsfalse %so that comments do not display

\ifcomments
\newcommand{\authornote}[3]{\textcolor{#1}{[#3 ---#2]}}
\newcommand{\todo}[1]{\textcolor{red}{[TODO: #1]}}
\else
\newcommand{\authornote}[3]{}
\newcommand{\todo}[1]{}
\fi

\newcommand{\wss}[1]{\authornote{magenta}{SS}{#1}} 
\newcommand{\plt}[1]{\authornote{cyan}{TPLT}{#1}} %For explanation of the template
\newcommand{\an}[1]{\authornote{cyan}{Author}{#1}}

%% Common Parts

\newcommand{\progname}{ProgName} % PUT YOUR PROGRAM NAME HERE
\newcommand{\authname}{Team \#, Team Name
\\ Student 1 name
\\ Student 2 name
\\ Student 3 name
\\ Student 4 name} % AUTHOR NAMES                  

\usepackage{hyperref}
    \hypersetup{colorlinks=true, linkcolor=blue, citecolor=blue, filecolor=blue,
                urlcolor=blue, unicode=false}
    \urlstyle{same}
                                


\begin{document}

\maketitle

\begin{table}[hp]
\caption{Revision History} \label{TblRevisionHistory}
\begin{tabularx}{\textwidth}{llX}
\toprule
\textbf{Date} & \textbf{Developer(s)} & \textbf{Change}\\
\midrule
Date1 & Name(s) & Description of changes\\
Date2 & Name(s) & Description of changes\\
... & ... & ...\\
\bottomrule
\end{tabularx}
\end{table}

\newpage{}

\wss{Put your introductory blurb here.  Often the blurb is a brief roadmap of
what is contained in the report.}

\wss{Additional information on the development plan can be found in the
\href{https://gitlab.cas.mcmaster.ca/courses/capstone/-/blob/main/Lectures/L02b_POCAndDevPlan/POCAndDevPlan.pdf?ref_type=heads}
{lecture slides}.}

\section{Confidential Information?}

\wss{State whether your project has confidential information from industry, or
not.  If there is confidential information, point to the agreement you have in
place.}

\wss{For most teams this section will just state that there is no confidential
information to protect.}
\section{IP to Protect}

\wss{State whether there is IP to protect.  If there is, point to the agreement.
All students who are working on a project that requires an IP agreement are also
required to sign the ``Intellectual Property Guide Acknowledgement.''}

\section{Copyright License}

\wss{What copyright license is your team adopting.  Point to the license in your
repo.}

\section{Team Meeting Plan}

\wss{How often will you meet? where?}

\wss{If the meeting is a physical location (not virtual), out of an abundance of
caution for safety reasons you shouldn't put the location online}

\wss{How often will you meet with your industry advisor?  when?  where?}

\wss{Will meetings be virtual?  At least some meetings should likely be
in-person.}

\wss{How will the meetings be structured?  There should be a chair for all meetings.  There should be an agenda for all meetings.}

\section{Team Communication Plan}

\wss{Issues on GitHub should be part of your communication plan.}

\section{Team Member Roles}

\wss{You should identify the types of roles you anticipate, like notetaker,
leader, meeting chair, reviewer.  Assigning specific people to those roles is
not necessary at this stage.  In a student team the role of the individuals will
likely change throughout the year.}

\section{Workflow Plan}

\begin{itemize}
	\item How will you be using git, including branches, pull request, etc.?
	\item How will you be managing issues, including template issues, issue
	classification, etc.?
  \item Use of CI/CD
\end{itemize}

\section{Project Decomposition and Scheduling}

\begin{itemize}
  \item How will you be using GitHub projects?
  \item Include a link to your GitHub project
\end{itemize}

\wss{How will the project be scheduled?  This is the big picture schedule, not
details. You will need to reproduce information that is in the course outline
for deadlines.}

\section{Proof of Concept Demonstration Plan}

What is the main risk, or risks, for the success of your project?  What will you
demonstrate during your proof of concept demonstration to convince yourself that
you will be able to overcome this risk?

\section{Expected Technology}

\wss{What programming language or languages do you expect to use?  What external
libraries?  What frameworks?  What technologies.  Are there major components of
the implementation that you expect you will implement, despite the existence of
libraries that provide the required functionality.  For projects with machine
learning, will you use pre-trained models, or be training your own model?  }

\wss{The implementation decisions can, and likely will, change over the course
of the project.  The initial documentation should be written in an abstract way;
it should be agnostic of the implementation choices, unless the implementation
choices are project constraints.  However, recording our initial thoughts on
implementation helps understand the challenge level and feasibility of a
project.  It may also help with early identification of areas where project
members will need to augment their training.}

Topics to discuss include the following:

\begin{itemize}
\item Specific programming language
\item Specific libraries
\item Pre-trained models
\item Specific linter tool (if appropriate)
\item Specific unit testing framework
\item Investigation of code coverage measuring tools
\item Specific plans for Continuous Integration (CI), or an explanation that CI
  is not being done
\item Specific performance measuring tools (like Valgrind), if
  appropriate
\item Tools you will likely be using?
\end{itemize}

\wss{git, GitHub and GitHub projects should be part of your technology.}

\section{Coding Standard}

\wss{What coding standard will you adopt?}

\newpage{}

\section*{Appendix --- Reflection}

\wss{Not required for CAS 741}

\textbf{Team}\\
3. Thankfully, our team had no disagreements during this deliverable and were all on the same page, so this will serve as a response to everyone's Q3.\\


\textbf{Andy Liang}
\begin{enumerate}
\item{Creating our development plan was crucial for several reasons specific to our ambitious project. First, our game involves complex technical challenges - procedural destructible environments, intelligent slime mold traversal, and physics-based interactions that could compound into performance issues. Without a clear plan, we could easily get lost trying to solve these problems simultaneously. The plan helped us identify our main risk early: ensuring the slime mold behavior works as intended while maintaining performance when combined with our voxel-based destructible environment. By recognizing this upfront, we can focus our proof of concept demonstration on exactly this integration challenge. Additionally, with our diverse team roles (Art Director, Character Artist, Environment Artist, Programmer, Music Director, Composer), coordination is essential. The plan establishes clear communication channels through Discord and GitHub, defines our workflow using pull requests and code reviews, and sets expectations for CI/CD implementation. Without this structure, our different specializations could easily work in isolation and create integration nightmares later. The scheduling aspect also forces us to think realistically about deliverable deadlines and break down our complex technical goals into manageable milestones. }
\item{Early Issue Detection: Given our concern about compounding errors between procedural systems and physics, automated testing can catch integration problems before they become major headaches.
Team Coordination: With multiple people working on different systems (art, code, audio), CI/CD ensures everyone's work integrates properly and nobody breaks someone else's features.
Code Quality Assurance: Our plan includes unit testing, security checks, and formatting verification, which is essential when working with C\# and Unit.y
Performance Monitoring: Since performance is a key risk with our procedural and physics systems, automated performance testing can flag issues early.}
\end{enumerate}

\textbf{BoWen Liu}
\begin{enumerate}
\item{Creating a development plan prior to starting the project is essential in aligning the team’s goal, and workflow in order to have an realistic and feasible starting point and roadmap on how to proceed in this project.}
\item{CI/CD improves traceability and accountability in one’s work both in terms of intra/inter team development as well as for upper management in a business context. The disadvantages to using CI/CD could be low quality of work to meet rigorous and sometimes unrealistic weekly milestones as well as adding unnecessary overhead when committing deliverables.}
\end{enumerate}

\textbf{Felix Hurst}
\begin{enumerate}
\item{Creating a development plan prior to starting a project ensures many aspects of proper organization. Everyone in the team knows what tasks they are responsible for, so different team members do not end up trying to do the same work, and know who to contact to ask questions about specific modules. The team has expectations set, including activity, quality, self-imposed deadlines, and meeting schedules. The team is ultimately guided by the development plan in nearly everything they do while working on the project. Without this kind of structure, team members would be spending a lot more time asking questions, causing delays in development. Or, they may underperform compared to the other team members’ internal expectations. It is important that everyone is on the same page to minimize the need for future questions and minimize the possibility of conflict within the team.}
\item{The advantages of using CI/CD include:
\begin{itemize}
\item{Pull requests could be verified to meet specified tests. This ensures poorly written code is not accepted into the repository.}
\item{New code could be automatically built into a testable version of the project, making it faster to test.}
\end{itemize}

The disadvantages of using CI/CD include:
\begin{itemize}
\item{It takes time to set it up and write tests, especially those that are intended to be universal across all newly accepted code.}
\item{It may slow down the process of merging pull requests for minor changes that don’t need extra testing.}
\end{itemize}
}
\end{enumerate}

\textbf{Marcos Hernandez-Rivero}
\begin{enumerate}
\item{Creating a development plan before starting a software engineering group project is essential because it provides a clear roadmap for the team, defining goals, scope, roles, and timelines to keep everyone aligned. It helps prevent confusion, overlap, or missed tasks by assigning responsibilities, establishes coding and documentation standards for consistency, and outlines milestones to manage time effectively. A development plan also anticipates risks to project success, and ideally sets strategies to address them.}
\item{CI/CD allows teams to integrate code frequently, and catch many errors early and automatically, which in the long run improves software quality and reduces the amount of bugs either on release or later down the production workflow. Some notable disadvantages though, are that it requires setting up and maintaining the CI/CD system, which can be time-consuming and/or confusing to many individuals. Additionally, any automated tests need to be thorough so that they act as a reliable tool to ensure code quality.}
\end{enumerate}

\begin{enumerate}
    \item Why is it important to create a development plan prior to starting the
    project?
    \item In your opinion, what are the advantages and disadvantages of using
    CI/CD?
    \item What disagreements did your group have in this deliverable, if any,
    and how did you resolve them?
\end{enumerate}

\newpage{}

\section*{Appendix --- Team Charter}

\wss{borrows from
\href{https://engineering.up.edu/industry_partnerships/files/team-charter.pdf}
{University of Portland Team Charter}}

\subsection*{External Goals}

\wss{What are your team's external goals for this project? These are not the
goals related to the functionality or quality fo the project.  These are the
goals on what the team wishes to achieve with the project.  Potential goals are
to win a prize at the Capstone EXPO, or to have something to talk about in
interviews, or to get an A+, etc.}

\subsection*{Attendance}

\subsubsection*{Expectations}

\wss{What are your team's expectations regarding meeting attendance (being on
time, leaving early, missing meetings, etc.)?}

\subsubsection*{Acceptable Excuse}

\wss{What constitutes an acceptable excuse for missing a meeting or a deadline?
What types of excuses will not be considered acceptable?}

\subsubsection*{In Case of Emergency}

\wss{What process will team members follow if they have an emergency and cannot
attend a team meeting or complete their individual work promised for a team
deliverable?}

\subsection*{Accountability and Teamwork}

\subsubsection*{Quality} 

\wss{What are your team's expectations regarding the quality
of team members' preparation for team meetings and the quality of the
deliverables that members bring to the team?}

\subsubsection*{Attitude}

\wss{What are your team's expectations regarding team members' ideas,
interactions with the team, cooperation, attitudes, and anything else regarding
team member contributions?  Do you want to introduce a code of conduct?  Do you
want a conflict resolution plan?  Can adopt existing codes of conduct.}

\subsubsection*{Stay on Track}

\wss{What methods will be used to keep the team on track? How will your team
ensure that members contribute as expected to the team and that the team
performs as expected? How will your team reward members who do well and manage
members whose performance is below expectations?  What are the consequences for
someone not contributing their fair share?}

\wss{You may wish to use the project management metrics collected for the TA and
instructor for this.}

\wss{You can set target metrics for attendance, commits, etc.  What are the
consequences if someone doesn't hit their targets?  Do they need to bring the
coffee to the next team meeting?  Does the team need to make an appointment with
their TA, or the instructor?  Are there incentives for reaching targets early?}

\subsubsection*{Team Building}

\wss{How will you build team cohesion (fun time, group rituals, etc.)? }

\subsubsection*{Decision Making} 

\wss{How will you make decisions in your group? Consensus?  Vote? How will you
handle disagreements? }

\end{document}