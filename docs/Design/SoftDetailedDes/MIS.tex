\documentclass[12pt, titlepage]{article}

\usepackage{amsmath, mathtools}

\usepackage[round]{natbib}
\usepackage{amsfonts}
\usepackage{amssymb}
\usepackage{graphicx}
\usepackage{colortbl}
\usepackage{xr}
\usepackage{hyperref}
\usepackage{longtable}
\usepackage{xfrac}
\usepackage{tabularx}
\usepackage{float}
\usepackage{siunitx}
\usepackage{booktabs}
\usepackage{multirow}
\usepackage[section]{placeins}
\usepackage{caption}
\usepackage{fullpage}

\hypersetup{
bookmarks=true,     % show bookmarks bar?
colorlinks=true,       % false: boxed links; true: colored links
linkcolor=red,          % color of internal links (change box color with linkbordercolor)
citecolor=blue,      % color of links to bibliography
filecolor=magenta,  % color of file links
urlcolor=cyan          % color of external links
}

\usepackage{array}

\externaldocument{../../SRS/SRS}

%% Comments

\usepackage{color}

\newif\ifcomments\commentstrue %displays comments
%\newif\ifcomments\commentsfalse %so that comments do not display

\ifcomments
\newcommand{\authornote}[3]{\textcolor{#1}{[#3 ---#2]}}
\newcommand{\todo}[1]{\textcolor{red}{[TODO: #1]}}
\else
\newcommand{\authornote}[3]{}
\newcommand{\todo}[1]{}
\fi

\newcommand{\wss}[1]{\authornote{magenta}{SS}{#1}} 
\newcommand{\plt}[1]{\authornote{cyan}{TPLT}{#1}} %For explanation of the template
\newcommand{\an}[1]{\authornote{cyan}{Author}{#1}}

%% Common Parts

\newcommand{\progname}{ProgName} % PUT YOUR PROGRAM NAME HERE
\newcommand{\authname}{Team \#, Team Name
\\ Student 1 name
\\ Student 2 name
\\ Student 3 name
\\ Student 4 name} % AUTHOR NAMES                  

\usepackage{hyperref}
    \hypersetup{colorlinks=true, linkcolor=blue, citecolor=blue, filecolor=blue,
                urlcolor=blue, unicode=false}
    \urlstyle{same}
                                


\begin{document}

\title{Module Interface Specification for \progname{}}

\author{\authname}

\date{\today}

\maketitle

\pagenumbering{roman}

\section{Revision History}

\begin{tabularx}{\textwidth}{p{3cm}p{2cm}X}
\toprule {\bf Date} & {\bf Version} & {\bf Notes}\\
\midrule
Date 1 & 1.0 & Notes\\
Date 2 & 1.1 & Notes\\
\bottomrule
\end{tabularx}

~\newpage

\section{Symbols, Abbreviations and Acronyms}

See SRS Documentation at \wss{give url}

\wss{Also add any additional symbols, abbreviations or acronyms}

\newpage

\tableofcontents

\newpage

\pagenumbering{arabic}

\section{Introduction}

The following document details the Module Interface Specifications for
\wss{Fill in your project name and description}

Complementary documents include the System Requirement Specifications
and Module Guide.  The full documentation and implementation can be
found at \url{...}.  \wss{provide the url for your repo}

\section{Notation}

\wss{You should describe your notation.  You can use what is below as
  a starting point.}

The structure of the MIS for modules comes from \citet{HoffmanAndStrooper1995},
with the addition that template modules have been adapted from
\cite{GhezziEtAl2003}.  The mathematical notation comes from Chapter 3 of
\citet{HoffmanAndStrooper1995}.  For instance, the symbol := is used for a
multiple assignment statement and conditional rules follow the form $(c_1
\Rightarrow r_1 | c_2 \Rightarrow r_2 | ... | c_n \Rightarrow r_n )$.

The following table summarizes the primitive data types used by \progname. 

\begin{center}
\renewcommand{\arraystretch}{1.2}
\noindent 
\begin{tabular}{l l p{7.5cm}} 
\toprule 
\textbf{Data Type} & \textbf{Notation} & \textbf{Description}\\ 
\midrule
character & char & a single symbol or digit\\
integer & $\mathbb{Z}$ & a number without a fractional component in (-$\infty$, $\infty$) \\
natural number & $\mathbb{N}$ & a number without a fractional component in [1, $\infty$) \\
real & $\mathbb{R}$ & any number in (-$\infty$, $\infty$)\\
\bottomrule
\end{tabular} 
\end{center}

\noindent
The specification of \progname \ uses some derived data types: sequences, strings, and
tuples. Sequences are lists filled with elements of the same data type. Strings
are sequences of characters. Tuples contain a list of values, potentially of
different types. In addition, \progname \ uses functions, which
are defined by the data types of their inputs and outputs. Local functions are
described by giving their type signature followed by their specification.

\section{Module Decomposition}

The following table is taken directly from the Module Guide document for this project.

\begin{table}[h!]
\centering
\begin{tabular}{p{0.3\textwidth} p{0.6\textwidth}}
\toprule
\textbf{Level 1} & \textbf{Level 2}\\
\midrule

{Hardware-Hiding} & ~ \\
\midrule

\multirow{7}{0.3\textwidth}{Behaviour-Hiding} & Input Parameters\\
& Output Format\\
& Output Verification\\
& Temperature ODEs\\
& Energy Equations\\ 
& Control Module\\
& Specification Parameters Module\\
\midrule

\multirow{3}{0.3\textwidth}{Software Decision} & {Sequence Data Structure}\\
& ODE Solver\\
& Plotting\\
\bottomrule

\end{tabular}
\caption{Module Hierarchy}
\label{TblMH}
\end{table}

\newpage
~\newpage

\section{MIS of \wss{Module Name}} \label{Module} \wss{Use labels for
  cross-referencing}

\wss{You can reference SRS labels, such as R\ref{R_Inputs}.}

\wss{It is also possible to use \LaTeX for hypperlinks to external documents.}

\subsection{Module}

\wss{Short name for the module}

\subsection{Uses}


\subsection{Syntax}

\subsubsection{Exported Constants}

\subsubsection{Exported Access Programs}

\begin{center}
\begin{tabular}{p{2cm} p{4cm} p{4cm} p{2cm}}
\hline
\textbf{Name} & \textbf{In} & \textbf{Out} & \textbf{Exceptions} \\
\hline
\wss{accessProg} & - & - & - \\
\hline
\end{tabular}
\end{center}

\subsection{Semantics}

\subsubsection{State Variables}

\wss{Not all modules will have state variables.  State variables give the module
  a memory.}

\subsubsection{Environment Variables}

\wss{This section is not necessary for all modules.  Its purpose is to capture
  when the module has external interaction with the environment, such as for a
  device driver, screen interface, keyboard, file, etc.}

\subsubsection{Assumptions}

\wss{Try to minimize assumptions and anticipate programmer errors via
  exceptions, but for practical purposes assumptions are sometimes appropriate.}

\subsubsection{Access Routine Semantics}

\noindent \wss{accessProg}():
\begin{itemize}
\item transition: \wss{if appropriate} 
\item output: \wss{if appropriate} 
\item exception: \wss{if appropriate} 
\end{itemize}

\wss{A module without environment variables or state variables is unlikely to
  have a state transition.  In this case a state transition can only occur if
  the module is changing the state of another module.}

\wss{Modules rarely have both a transition and an output.  In most cases you
  will have one or the other.}

\subsubsection{Local Functions}

\wss{As appropriate} \wss{These functions are for the purpose of specification.
  They are not necessarily something that is going to be implemented
  explicitly.  Even if they are implemented, they are not exported; they only
  have local scope.}

\newpage

\bibliographystyle {plainnat}
\bibliography {../../../refs/References}

\newpage

\section{Appendix} \label{Appendix}

\wss{Extra information if required}

\newpage{}

\section*{Appendix --- Reflection}

\wss{Not required for CAS 741 projects}

The information in this section will be used to evaluate the team members on the
graduate attribute of Problem Analysis and Design.

\textbf{Team}\\
3. Thankfully, our team had no disagreements during this deliverable and were all on the same page, so this will serve as a response to everyone's Q3.\\


\textbf{Andy Liang}
\begin{enumerate}
\item{Creating our development plan was crucial for several reasons specific to our ambitious project. First, our game involves complex technical challenges - procedural destructible environments, intelligent slime mold traversal, and physics-based interactions that could compound into performance issues. Without a clear plan, we could easily get lost trying to solve these problems simultaneously. The plan helped us identify our main risk early: ensuring the slime mold behavior works as intended while maintaining performance when combined with our voxel-based destructible environment. By recognizing this upfront, we can focus our proof of concept demonstration on exactly this integration challenge. Additionally, with our diverse team roles (Art Director, Character Artist, Environment Artist, Programmer, Music Director, Composer), coordination is essential. The plan establishes clear communication channels through Discord and GitHub, defines our workflow using pull requests and code reviews, and sets expectations for CI/CD implementation. Without this structure, our different specializations could easily work in isolation and create integration nightmares later. The scheduling aspect also forces us to think realistically about deliverable deadlines and break down our complex technical goals into manageable milestones. }
\item{Early Issue Detection: Given our concern about compounding errors between procedural systems and physics, automated testing can catch integration problems before they become major headaches.
Team Coordination: With multiple people working on different systems (art, code, audio), CI/CD ensures everyone's work integrates properly and nobody breaks someone else's features.
Code Quality Assurance: Our plan includes unit testing, security checks, and formatting verification, which is essential when working with C\# and Unit.y
Performance Monitoring: Since performance is a key risk with our procedural and physics systems, automated performance testing can flag issues early.}
\end{enumerate}

\textbf{BoWen Liu}
\begin{enumerate}
\item{Creating a development plan prior to starting the project is essential in aligning the team’s goal, and workflow in order to have an realistic and feasible starting point and roadmap on how to proceed in this project.}
\item{CI/CD improves traceability and accountability in one’s work both in terms of intra/inter team development as well as for upper management in a business context. The disadvantages to using CI/CD could be low quality of work to meet rigorous and sometimes unrealistic weekly milestones as well as adding unnecessary overhead when committing deliverables.}
\end{enumerate}

\textbf{Felix Hurst}
\begin{enumerate}
\item{Creating a development plan prior to starting a project ensures many aspects of proper organization. Everyone in the team knows what tasks they are responsible for, so different team members do not end up trying to do the same work, and know who to contact to ask questions about specific modules. The team has expectations set, including activity, quality, self-imposed deadlines, and meeting schedules. The team is ultimately guided by the development plan in nearly everything they do while working on the project. Without this kind of structure, team members would be spending a lot more time asking questions, causing delays in development. Or, they may underperform compared to the other team members’ internal expectations. It is important that everyone is on the same page to minimize the need for future questions and minimize the possibility of conflict within the team.}
\item{The advantages of using CI/CD include:
\begin{itemize}
\item{Pull requests could be verified to meet specified tests. This ensures poorly written code is not accepted into the repository.}
\item{New code could be automatically built into a testable version of the project, making it faster to test.}
\end{itemize}

The disadvantages of using CI/CD include:
\begin{itemize}
\item{It takes time to set it up and write tests, especially those that are intended to be universal across all newly accepted code.}
\item{It may slow down the process of merging pull requests for minor changes that don’t need extra testing.}
\end{itemize}
}
\end{enumerate}

\textbf{Marcos Hernandez-Rivero}
\begin{enumerate}
\item{Creating a development plan before starting a software engineering group project is essential because it provides a clear roadmap for the team, defining goals, scope, roles, and timelines to keep everyone aligned. It helps prevent confusion, overlap, or missed tasks by assigning responsibilities, establishes coding and documentation standards for consistency, and outlines milestones to manage time effectively. A development plan also anticipates risks to project success, and ideally sets strategies to address them.}
\item{CI/CD allows teams to integrate code frequently, and catch many errors early and automatically, which in the long run improves software quality and reduces the amount of bugs either on release or later down the production workflow. Some notable disadvantages though, are that it requires setting up and maintaining the CI/CD system, which can be time-consuming and/or confusing to many individuals. Additionally, any automated tests need to be thorough so that they act as a reliable tool to ensure code quality.}
\end{enumerate}

\begin{enumerate}
  \item What went well while writing this deliverable? 
  \item What pain points did you experience during this deliverable, and how
    did you resolve them?
  \item Which of your design decisions stemmed from speaking to your client(s)
  or a proxy (e.g. your peers, stakeholders, potential users)? For those that
  were not, why, and where did they come from?
  \item While creating the design doc, what parts of your other documents (e.g.
  requirements, hazard analysis, etc), it any, needed to be changed, and why?
  \item What are the limitations of your solution?  Put another way, given
  unlimited resources, what could you do to make the project better? (LO\_ProbSolutions)
  \item Give a brief overview of other design solutions you considered.  What
  are the benefits and tradeoffs of those other designs compared with the chosen
  design?  From all the potential options, why did you select the documented design?
  (LO\_Explores)
\end{enumerate}


\end{document}